\pdfoutput=1
\documentclass[preview]{standalone}

\begin{document}

\begin{center}
estudiante(\underline{legajo}, dni, nombre, apellido)\\
\vspace{.35cm}
\begin{tabular}{| l | c | l | l | }\hline			
	legajo & dni & nombre & apellido \\\hline			
	1 & 25.333.213 & Juan & P\'erez \\
	2 & 27.341.223 & Mariana & Na\'on\\
	3 & 32.341.784 & Pilar & Gamboa \\
	4 & 29.311.665 & Juan & P\'erez \\\hline	
\end{tabular}
%\vspace{.35cm}
\vspace{0.7cm}

\textit{¿Qué se obtiene con esta combinación de operaciones?}
$${\Pi_{\textsubscript{dni, apellido}} (\sigma_{\textsubscript{nombre = 'Juan'}} \ \textrm{estudiante} )}$$
\vspace{0.2cm}

\textit{¿Tiene sentido combinar las operaciones en otro orden?}
$${\sigma_{\textsubscript{nombre = 'Juan'}} (\Pi_{\textsubscript{dni, apellido}} \ \textrm{estudiante} )}$$
\vspace{.35cm}
\end{center}

\end{document}
